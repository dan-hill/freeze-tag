\documentclass[twoside]{article}

%-------------------------------------------------------------------------------

\usepackage{amsthm}
\usepackage[sc]{mathpazo} 	% palatino font
\usepackage[T1]{fontenc} 	% 8-bit encoding; 256 glyphs
\linespread{1.05} 			% line spacing
\usepackage{microtype} 		% slightly tweak font spacing
\PassOptionsToPackage{usenames,dvipsnames,svgnames}{xcolor}  
\usepackage{tikz}
\usetikzlibrary{arrows,positioning,automata}
\usepackage[hmarginratio=1:1,top=32mm,columnsep=20pt]{geometry} 	% Document margins
\usepackage{multicol} 												% two-column layout
\usepackage[hang, small,labelfont=bf,up,textfont=it,up]{caption} 	% float captions
\usepackage{booktabs} 		% Horizontal rules in tables
\usepackage{float} 			% allows tables/figures in the multi-column environment
\usepackage{hyperref} 		% hyperlinks in the PDF
\usepackage{subfig}			% figures inside of figure environment
\usepackage{lettrine} 		% large first letter
\usepackage{paralist} 		% compactitem environment
\usepackage{abstract} 		% abstract customization
\renewcommand{\abstractnamefont}{\normalfont\bfseries} 		% bold abstract text
\renewcommand{\abstracttextfont}{\normalfont\small\itshape} % small italic abstract

\newtheorem{theorem}{Theorem}

\usepackage{titlesec} 							% title customization
\renewcommand\thesection{\Roman{section}} 		% roman numerals for the sections
\renewcommand\thesubsection{\Roman{subsection}} % roman numerals for subsections
\titleformat{\section}[block]{\large\scshape\centering}{\thesection.}{1em}{} % Change the look of the section titles
\titleformat{\subsection}[block]{\large}{\thesubsection.}{1em}{} % Change the look of the section titles

\usepackage{algorithm}
\usepackage{algorithmicx}
\usepackage[noend]{algpseudocode}
\usepackage{fontspec}
\newcommand*\Let[2]{\State #1 $\gets$ #2}

\algrenewcommand\algorithmicrequire{\textbf{Precondition:}}
\algrenewcommand\algorithmicensure{\textbf{Postcondition:}}

\usepackage{fancyhdr} 	% Headers and footers
\pagestyle{fancy} 		% All pages have headers and footers
\fancyhead{} 			% Blank out the default header
\fancyfoot{} 			% Blank out the default footer
\fancyhead[C]{Algorithms for the Online Time Dependant Freeze-Tag Problem $\bullet$ July 2015 $\bullet$ Dan Hill} % Custom header text
\fancyfoot[RO,LE]{\thepage} % Custom footer text		% includes and document setup

%-------------------------------------------------------------------------------

\newcommand{\articletitle}{Algorithms for the Online Time Dependant Freeze-Tag Problem}
\newcommand{\name}{Dan Hill}
\newcommand{\university}{Cameron University}
\newcommand{\email}{mail@danhill.us}


\title{\vspace{-15mm}\fontsize{24pt}{10pt}\selectfont\textbf{\articletitle}} % Article title

\author{\large\textsc{\name} \\[2mm] 		% Your name
\normalsize \university \\		 			% Your institution
\normalsize \href{mailto:\email}{\email} 	% Your email address
\vspace{-5mm}}

\date{}		% title formatting

%-------------------------------------------------------------------------------

\begin{document}
\maketitle % Insert title
\thispagestyle{fancy} % All pages have headers and footers
%----------------------------------------------------------------------------------------
%	ABSTRACT
%----------------------------------------------------------------------------------------

\begin{abstract}

The abstract lives here.

\end{abstract}	% abstract

%----------------------------------------------------------------------------------------
%	ARTICLE CONTENTS
%----------------------------------------------------------------------------------------

\begin{multicols}{2} % Two-column layout throughout the main article text


%----------------------------------------------------------------------------------------
% Introduction
%----------------------------------------------------------------------------------------
\section{Introduction}
The Freeze-Tag Problem (FTP) \cite{FTP.0} is a problem in the the field robotics in which a strategy to awaken a swarm must be found. In the original problem, we are given a swarm sleeping robots and a single awake robot. Sleeping robots are awakened when touched by an awake robot.

In this paper we examine a variation of FTP \cite{FTP.1} in which each robot has a release time associated with it. Awake robots do not become aware of a sleeping robot's existence until it's release time has been met. We call this variation the Online Time Dependant Freeze-Tag Problem ( OTDFTP). 
\paragraph{Related Work.}
Hammer et al. \cite{FTP.1} found the Offline Time Dependant Freeze Tag Problem has a lower bound of $7/3 - \epsilon$, for any $\epsilon > 0$. 
\paragraph{Preliminaries.}
Let $R = \{r_0, r_1, r_2, \dots , r_n \} \subset M$ be the set of $n$ robots in some continuous metric space $M$. $M$ is a $d$-dimensional Euclidean space with distances measured according to an $L_p$ metric.

Unless otherwise stated, the robot $r_0$ is the \textit{source robot} and is the only robot that is not in sleep mode. Each robot has 3 modes. In awake mode, a robot is fully functional and free to move. In sleep mode a robot is completely inactive and unable to be activated even if touched by an awake robot. Once the release time of a robot has been reached it enters \textit{wait mode} and awake robots can now sense it's position. In wait mode a robot is available for activation.

Let $\sigma = (r, v)_1, (r, v)_2, \dots (r, v)_m$ be a set of locations of robots ordered by their release time.
\paragraph{Summary of Results.}
We find that the competitive ratio of the Nearest Neighbor heuristic is $5/2$.
\section{Wake Up Strategies}
\subsection{Nearest Neighbor Algorithm}
To keep robots from traveling in a pack we utilize claims in which when a robot chooses a target, it will claim that target and no other robots will target that robot. \cite{FTP.2}
\begin{algorithm}[H]
  \caption{Returns the nearest unclaimed sleeping robot.}
  \begin{algorithmic}
    \Require{$A$ is the set of sleeping robots and $t$ be the robot's current target.}
    \Statex
    \Function{Nearest Neighbor}{$A$, $t$}
		\If{$t \neq$ NULL}
	        \State \Return $t$
        \EndIf{}
        
        \If{$A.size = 0$}
	        \State \Return NULL
	    \EndIf
	    
        \Let{$m$}{NULL}
        
        \For{$robot$ in $A$}
	        \If{$\neg robot.claimed$}
	        
				\If{$m =$ NULL}
					\Let{$m$}{$robot$}
				\Else
					\Let{$a$}{dist($self$, $robot$)}
			        \Let{$b$}{dist($self$, $m$)}
			        \If{$a < b$}
				        \Let{$m$}{$robot$}
			        \EndIf
			    \EndIf
			\EndIf
        \EndFor
        \State \Return $m$
    \EndFunction
  \end{algorithmic}
\end{algorithm}

\begin{theorem}
For any $\epsilon > 0$, there exists an instance of the OTDFTP for which the Nearest Neighbor Algorithm results in a makespan less than $\frac{5}{2} - \epsilon$ times optimal.
\end{theorem}
\begin{proof}

\begin{figure}[H]
\tikzset{location/.style={circle, draw=blue, fill=white, thick, minimum size=.5cm}}
\tikzset{active_robot/.style={circle, fill=black, minimum size=.5cm}}
\centering
	\begin{tikzpicture}[scale=1, transform shape, label={t=0}]
		\node[location, label=below left:$p_0$]	(0)										{};
		\node[location, label=below left:$p_1$]	(1) [above left = 1.5cm and .5cm of 0]	{};
		\node[location, label=below right:$p_2$](2) [above right = 1.5cm and .5cm of 0]	{};
		\node[location, label=below:$p_3$]		(3) [below = 3cm of 0]{};
	
		\tikzset{mystyle/.style={-,double=black}} 
		\tikzset{every node/.style={fill=white, scale=1}} 
		\path 
			(0)	edge [mystyle]					(1)
			(0) edge [mystyle]	node	{$3$} 	(2)
			(0) edge [mystyle] 	node   	{$6$} 	(3)
			(1) edge [mystyle] 	node   	{$1$} 	(2);
	\end{tikzpicture}
	\caption{}
	\label{fig:1}
	
\end{figure}

Let $G = {V, E}$ be the graph in Figure \ref{fig:1}. The source robot $r_0$ activates at vertex  $v_0$. At time $t = 0$ robot $r_1$ enters the sleeping state and is immediately awakened by $r_0$. At $t = 3$ $r_2$ enters sleeping state at $v_1$ and $r_3$ enters sleeping state at $v_2$. $r_0$ claims $r_2$ and begins moving down edge $(v_0, v_1)$. $r_1$ claims $r_3$ and begins moving down edge $(v_0, v_2)$. $r_0$ and $r_1$ arrive and awaken their respective targets at $t = 6$. At the same time $r_4$ enters sleeping state at $v_3$ and is claimed by $r_0$. $r_0$ travels down $(v_1, v_0)$ to $v_0$ then travels down $(v_0, v_3)$. $r_0$ arrives at $v_0$ at $t = 15$.

In the optimum solution, the source robot $r_0$ still starts at $v_0$ and awakens $r_1$ at $t = 0$. $r_0$ immediately moves down $(v_0, v_3)$ and $r_1$ moves down $(v_0, v_1)$. At $t = 3$, $r_1$ arrives at $v_1$ as $r_2$ and $r_3$ are entering sleeping state. $r_1$ immediately claims and awakens $r_2$ then claims $r_2$ and begins moving down $(v_1, v_2)$. $r_1$ arrives at $v_2$ and awakens $r_3$ at $t = 4$. At $t = 6$ $r_0$ arrives at $v_3$ as $r_4$ is entering sleep mode and immediately awakens it. 

This gives us a makespan of $15$ for the Nearest Neighbor algorithm and a makespan of 6 for the optimal solution to Figure \ref{fig:1}. This gives us a competitive ratio of  $\frac{5}{2}.$
\end{proof}


\subsection{Density Based Algorithm}

\subsection{Sibling Based Algorithm}





%----------------------------------------------------------------------------------------
%	REFERENCE LIST
%----------------------------------------------------------------------------------------

\bibliography{bibliography}{}
\bibliographystyle{plain}

%----------------------------------------------------------------------------------------

\end{multicols}

\end{document}
\grid
